\documentclass[aspectratio=169]{beamer}
\mode<presentation>
%\usetheme{Warsaw}
%\usetheme{Goettingen}
\usetheme{Hannover}
%\useoutertheme{default}

%\useoutertheme{infolines}
\useoutertheme{sidebar}
\usecolortheme{dolphin}


\setbeamersize{sidebar width left=0pt} % to remove the sidebar
\beamertemplatenavigationsymbolsempty % To remove the navigation symbols on the bottom right.
\setbeamersize{text margin left=10mm,text margin right=10mm} % Specify margins

\usepackage{amsmath}
\usepackage{amssymb}
\usepackage{listings}
\usepackage{enumerate}
\usepackage{hyperref}
\hypersetup{
    colorlinks=true,
    linkcolor=blue,
    filecolor=magenta,      
    urlcolor=cyan,
}
 
\urlstyle{same}

%some bold math symbosl
\newcommand{\Cov}{\mathrm{Cov}}
\newcommand{\Var}{\mathrm{Var}}
\newcommand{\brho}{\boldsymbol{\rho}}
\newcommand{\bSigma}{\boldsymbol{\Sigma}}
\newcommand{\btheta}{\boldsymbol{\theta}}
\newcommand{\bbeta}{\boldsymbol{\beta}}
\newcommand{\bmu}{\boldsymbol{\mu}}
\newcommand{\bW}{\mathbf{W}}
\newcommand{\one}{\mathbf{1}}
\newcommand{\bH}{\mathbf{H}}
\newcommand{\by}{\mathbf{y}}
\newcommand{\bolde}{\mathbf{e}}
\newcommand{\bx}{\mathbf{x}}

\newcommand{\cpp}[1]{\texttt{#1}}

%--------------------------------------------------
\providecommand{\abs}[1]{\lvert#1\rvert}
\providecommand{\norm}[1]{\lVert#1\rVert}
\providecommand{\Blue}[1]{\textcolor{blue}{#1}}
\providecommand{\Red}[1]{\textcolor{red}{#1}}
\newcommand{\celsius}{\ensuremath{^\circ}C}
\newcommand\thfore{\mathord{\therefore}\,}
%------------------------------------------------------------------

\title{Lecture 23. Linear Nonhomogeneous Recurrence Relations}
%\author{\includegraphics[width=.5\textwidth,height=.5\textheight]{lecture4-fig0.png}}

\date{ }
%------------------------------------------------------------------


\begin{document}

\frame[plain]{\titlepage}



\begin{frame}[plain]{}

 {\bf  Definition 23.1}. A \Blue{linear nonhomogeneous recurrence relation}
   of degree $k$ with constant coefficients is a recurrence relation of the form 
   \[ \Blue{ a_n = r_1a_{n-1} + r_{2}a_{n-2}+\cdots + r_ka_{n-k} + F(n) } 
   \]
   where $r_1, r_2, ..., r_k$ are real numbers  ($r_k\neq 0$) with $k<n$
    and $F(n)$ is a function not identically zero depending only on $n$.
    The same recurrence with
 $F(n)$ omitted is called the \Blue{associated homogeneous recurrence relation}.
 \medskip
 \pause
 
  {\bf Algorithm} of solving a nonhomogeneous recurrence relation:
  
 \begin{enumerate}
  \item Find the solution to the associated homogeneous recurrence relation $\Blue{a_n^{(h)} }$.
  \item Find \Red{a particular solution} of the nonhomogeneous linear recurrence relation $\Blue{a_n^{(p)} }$. 
  \item $\Blue{a_n^{(h)}+a_n^{(p)} }$ forms the solution to the nonhomogeneous recurrence relation:
       \[ \Blue{a_n = a_n^{(h)}+a_n^{(p)} }. 
       \]
  \end{enumerate}

 
\end{frame}

\begin{frame}[plain]{}

 {\bf Example 23.2}. Find all solutions of the recurrence relation $a_n = 3a_{n-1} + 2n$ with $a_1=3$.
 \smallskip
 \pause
 
 Answer: $a_n = -n-\frac{3}{2}+\frac{11}{6}3^n$.
 
 \medskip
 
 {\bf Problem 23.3}. Find all solutions of the recurrence relation $a_n = 5a_{n-1}-6a_{n-2}+7^n$.

 \vspace{1.2in}
 
\end{frame}


\begin{frame}[plain]{}

 {\bf  Theorem 23.4}. Suppose that ${a_n}$ satisfies the linear nonhomogeneous recurrence
    \[ \Blue{ a_n = r_1a_{n-1} + r_{2}a_{n-2}+\cdots + r_ka_{n-k} + F(n)}, 
   \]
   where $r_1, r_2, ..., r_k$ are real numbers  and
   \[ \Blue{ F(n) = (b_tn^t+b_{t-1}n^{t-1}+\cdots + b_1n+b_0)s^n},\]
   where $b_0,...b_t$ and $s$ are real numbers.
   \begin{itemize}
     \item[(a)] When $s$ is \Red{not a root of the characteristic equation} of the associated linear homogeneous
            recurrence, there is a particular solution of the form
             \[ \Blue{ a_n^{(p)} = (p_tn^t+p_{t-1}n^{t-1}+\cdots + p_1n+p_0)s^n},\]
      \item[(b)] When $s$ is a root of this characteristic equation and its multiplicity is $m$, 
              there is a particular solution of the form
               \[ \Blue{ a_n^{(p)} = n^m (p_tn^t+p_{t-1}n^{t-1}+\cdots + p_1n+p_0)s^n},\]
  \end{itemize}
   
    
\end{frame}

\begin{frame}[plain]{}

{\bf Example 23.5}. What form does a particular solution of the linear nonhomogeneous recurrence
  \[ a_n = 6a_{n-1}-9a_{n-2}+F(n) \]
  have when $F(n) = 3^n$, $F(n) = n3^n$, $F(n) = n^2 2^n$, and $F(n) = (n^2+1)3^n$?  
  \smallskip
  
  \begin{center}
  \begin{tabular}{|c|c|} \hline 
   $F(n)$ & $a_n^{(p)}$  \\ \hline
     $3^n$ &  $n^2p_0 3^n$  \\ \hline
      $n3^n$ &  $n^2(p_1n+p_0)3^n$  \\ \hline
          $n^2 2^n$ &  $(p_2n^2+p_1n+p_0)2^n$  \\ \hline
            $n^2 3^n$ &  $n^2(p_2n^2+p_1n+p_0)3^n$  \\ \hline
  \end{tabular}
  \end{center}
\pause

\medskip

{\bf Problem 23.6 on WeBWork}. Solve the linear nonhomogeneous recurrence relation
\[
\Blue{a_n = 2a_{n-1} - a_{n-2} + 2^n + 2}
\]
with initial conditions \(\Red{a_1 = 7}\) and \(\Red{a_2 = 19}\). \\
  (Test Time Limit on WeBWork: 15 min)

\end{frame}


\end{document}
\documentclass[aspectratio=169]{beamer}
\mode<presentation>
%\usetheme{Warsaw}
%\usetheme{Goettingen}
\usetheme{Hannover}
%\useoutertheme{default}

%\useoutertheme{infolines}
\useoutertheme{sidebar}
\usecolortheme{dolphin}


\setbeamersize{sidebar width left=0pt} % to remove the sidebar
\beamertemplatenavigationsymbolsempty % To remove the navigation symbols on the bottom right.
\setbeamersize{text margin left=10mm,text margin right=10mm} % Specify margins

\usepackage{amsmath}
\usepackage{amssymb}
\usepackage{listings}
\usepackage{enumerate}
\usepackage{hyperref}
\hypersetup{
    colorlinks=true,
    linkcolor=blue,
    filecolor=magenta,      
    urlcolor=cyan,
}
 
\urlstyle{same}

%some bold math symbosl
\newcommand{\Cov}{\mathrm{Cov}}
\newcommand{\Var}{\mathrm{Var}}
\newcommand{\brho}{\boldsymbol{\rho}}
\newcommand{\bSigma}{\boldsymbol{\Sigma}}
\newcommand{\btheta}{\boldsymbol{\theta}}
\newcommand{\bbeta}{\boldsymbol{\beta}}
\newcommand{\bmu}{\boldsymbol{\mu}}
\newcommand{\bW}{\mathbf{W}}
\newcommand{\one}{\mathbf{1}}
\newcommand{\bH}{\mathbf{H}}
\newcommand{\by}{\mathbf{y}}
\newcommand{\bolde}{\mathbf{e}}
\newcommand{\bx}{\mathbf{x}}

\newcommand{\cpp}[1]{\texttt{#1}}

%--------------------------------------------------
\providecommand{\abs}[1]{\lvert#1\rvert}
\providecommand{\norm}[1]{\lVert#1\rVert}
\providecommand{\Blue}[1]{\textcolor{blue}{#1}}
\providecommand{\Red}[1]{\textcolor{red}{#1}}
\newcommand{\celsius}{\ensuremath{^\circ}C}
\newcommand\thfore{\mathord{\therefore}\,}
%------------------------------------------------------------------

\title{Lecture 21. Solving Recurrence Relations }
%\author{\includegraphics[width=.5\textwidth,height=.5\textheight]{lecture4-fig0.png}}

\date{ }
%------------------------------------------------------------------


\begin{document}

\frame[plain]{\titlepage}

\begin{frame}[plain]{Recursive Thinking}

In the previous lecture note, we have seen some examples 
how to think recursively about a problem by describing it
with a recurrence relation.
 Remember that any recurrence relation has two parts: a {\bf base case}
that describes some initial conditions, and a {\bf recursive case}
 that describes a {\bf future value in
terms of previous values}. 
Armed with this way of thinking, we can model other problems using
recurrence relations.
\medskip
\pause

{\bf Example 20.1}. Ahmed lends money at outrageous rates of interest. 
He demands to be paid
     10\% interest \emph{per week} on a loan, compounded weekly.
     Suppose you borrow 500 Dhs from him.
      Let $M(n)$ = the money you owed at $n$-th week.
     \begin{itemize}
       \item[(a)]  Find the recurrence relation for $M(n)$. 
       \item[(b)] If you wait four weeks to pay him back, how much will you owe?
     \end{itemize}
%Essential, p176, Example 3.3
\pause

\smallskip

{\bf Answer:} 
\[ M(n) = \left\{ \begin{array}{ccc}
                       500 &\mbox{if}& n = 0\\
                       1.1\,M(n-1) &\mbox{if}& n>0
                      \end{array} \right. 
              \ M(4) = \$ 732.05
    \]

\end{frame}

\begin{frame}[plain]{}

  {\bf Practice 20.2}. Find a closed-form solution 
  for the recurrence relation from
  \[ M(n) = \left\{ \begin{array}{ccc}
                       500 &\mbox{if}& n = 0\\
                       1.1\,M(n-1) &\mbox{if}& n>0
                      \end{array} \right. 
    \]
    
    \pause
    
 {\bf Answer:} $M(n) = 500 (1.1)^n$ %Using the backtracking method
 
 \vspace{1in}
 
\end{frame}


\begin{frame}[plain]{Polynomial sequences: Using differences}

 Finding a closed-form solution for $P(n)$ can be thought of as guessing a formula
    for the sequence $P(0)$, $P(1)$, $P(2)$, ....  \pause
    \medskip
    
 {\bf One Idea}.  If we knew that  
     \[ P(n) = \ \mbox{a polynomial function of }\ n
       % = a_m n^m + a_{m-1}n^{m-1},\cdots + a_1n + a_0 
      \]
     then we can detect such a sequence by looking at the differences 
     between terms.
\pause

 Given any sequence,
     \[ a_0, a_1, a_2, ..., a_{n-1}, a_n \]
  the differences 
   \[ a_1-a_0, a_2-a_1, a_3-a_2, ..., a_n-a_{n-1} \]
   form another sequence, called the \Blue{sequence of differences}.
 

\end{frame}

\begin{frame}[plain]{}

  Given any sequence,
  \[ a_0, a_1, a_2, ..., a_{n-1}, a_n \]
  the differences 
  \[ a_1-a_0, a_2-a_1, a_3-a_2, ..., a_n-a_{n-1} \]
  form another sequence, called the \Blue{sequence of differences}.

 \begin{itemize}
  \item A linear sequence \Blue{$a_n = An+B$} will have a constant sequence of differences (because a line has constant slop).
      \item A quadratic sequence \Blue{$a_n = An^2+Bn+C$} will have a linear sequence of differences.
      \item a cubic sequence \Blue{$a_n = An^3+Bn^2+Cn+D$}  will have a quadratic sequence of differences, etc.
      \item If we eventually end up with a constant sequence, then the original sequence
    is given by a polynomial function. 
      \item The degree of the conjectured polynomial
       is the number of times we had to calculate the 
    sequence of differences.
 \end{itemize}

\end{frame}

\begin{frame}[plain]{}

{\bf Example 20.3}. Find a closed-form solution $f(n)$ for the recurrence relation 
    \[ H(n) = \left\{ \begin{array}{ccc}
                       1 &\mbox{if}& n = 1\\
                       H(n-1)+6n-6 &\mbox{if}& n>1
                      \end{array} \right. 
    \]
    %Essential, p187,  Example 3.9
  \medskip
  
  \pause
  
  {\bf Answer}. $H(n) = 3n^2-3n+1$ is a \Red{good candidate}
   for a closed-form solution.
   
   \medskip
   
  {\bf Remark}. The result of these procedures is still only a guess.
   To be sure that our guess is right, we need to prove that the formula 
   matches the recurrence relation \Red{for all $n$}. \pause
\medskip

{\bf Practice 20.4}. Solve the recurrence relation $a_n = a_{n-1}+n$
  with initial term $a_0=4$. \pause
  
  \medskip
  
  {\bf Answer}. $a_n = \frac{n(n+1)}{2}+4$ for $n\geq 0$.
  
 \vspace{1in}

\end{frame}


\end{document}
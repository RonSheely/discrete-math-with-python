\documentclass[aspectratio=169]{beamer}
\mode<presentation>
%\usetheme{Warsaw}
%\usetheme{Goettingen}
\usetheme{Hannover}
%\useoutertheme{default}

%\useoutertheme{infolines}
\useoutertheme{sidebar}
\usecolortheme{dolphin}


\setbeamersize{sidebar width left=0pt} % to remove the sidebar
\beamertemplatenavigationsymbolsempty % To remove the navigation symbols on the bottom right.
\setbeamersize{text margin left=10mm,text margin right=10mm} % Specify margins

\usepackage{amsmath}
\usepackage{amssymb}
\usepackage{listings}
\usepackage{enumerate}
\usepackage{hyperref}
\hypersetup{
    colorlinks=true,
    linkcolor=blue,
    filecolor=magenta,      
    urlcolor=cyan,
}
 
\urlstyle{same}

%some bold math symbosl
\newcommand{\Cov}{\mathrm{Cov}}
\newcommand{\Var}{\mathrm{Var}}
\newcommand{\brho}{\boldsymbol{\rho}}
\newcommand{\bSigma}{\boldsymbol{\Sigma}}
\newcommand{\btheta}{\boldsymbol{\theta}}
\newcommand{\bbeta}{\boldsymbol{\beta}}
\newcommand{\bmu}{\boldsymbol{\mu}}
\newcommand{\bW}{\mathbf{W}}
\newcommand{\one}{\mathbf{1}}
\newcommand{\bH}{\mathbf{H}}
\newcommand{\by}{\mathbf{y}}
\newcommand{\bolde}{\mathbf{e}}
\newcommand{\bx}{\mathbf{x}}

\newcommand{\cpp}[1]{\texttt{#1}}

%--------------------------------------------------
\providecommand{\abs}[1]{\lvert#1\rvert}
\providecommand{\norm}[1]{\lVert#1\rVert}
\providecommand{\Blue}[1]{\textcolor{blue}{#1}}
\providecommand{\Red}[1]{\textcolor{red}{#1}}
\newcommand{\celsius}{\ensuremath{^\circ}C}
\newcommand\thfore{\mathord{\therefore}\,}
%------------------------------------------------------------------

\title{Lecture 22. Linear Homogeneous Recurrence Relations}
%\author{\includegraphics[width=.5\textwidth,height=.5\textheight]{lecture4-fig0.png}}

\date{ }
%------------------------------------------------------------------


\begin{document}

\frame[plain]{\titlepage}



\begin{frame}[plain]{}

 {\bf  Definition 22.1}. A \Blue{linear homogeneous recurrence relation
   of degree $k$ with constant coefficients} is a recurrence relation of the form 
   \[ \Blue{a_n = r_1a_{n-1} + r_{2}a_{n-2}+\cdots + r_ka_{n-k} } 
   \]
   where $r_1, r_2, ..., r_k$ are real numbers and $r_k\neq 0$ with $k<n$. 
   This recurrence includes $k$ initial conditions, 
   $a_0 = \alpha_0, a_1 = \alpha_1,..., a_k = \alpha_k$. \pause
   \begin{itemize}
     \item {\bf Linear}: Each term $a_j$ in the relation appears with an exponent of 1.\pause
     \item {\bf Homogeneous}: The relation contains no additional terms that directly 
     depend on $n$  (i.e., no external or independent functions of $n$). \pause
     \item {\bf Degree $k$}: Each term $a_n$ is defined by the previous $k$ terms in the sequence. \pause
   \end{itemize}
 
 {\bf Example 22.2}. Linear and/or Homogeneous?
 {\small
  \[ (a)\ M_n = (1.1)M_{n-1}\ \ (b)\ F_n = F_{n-1} + F_{n-2} \ \ 
     (c)\ a_n = a_{n-1}+a_{n-2}^2\ \ 
      (d)\ h_n = 2h_{n-1}+1\ \  
    % (e)\ b_n = n\,b_{n-1}
  \] 
  
  \pause
  
  (a) linear homogeneous recurrence relation of degree 1,\
  (b) linear homogeneous recurrence relation of degree 2,\
  (c) not linear,\
  (d) not homogeneous,\
  %(e) coefficients are not constants.
  }
  
\end{frame}

\begin{frame}[plain]{}

\begin{itemize}
  \item The basic approach is to look for solutions of the form
       \Blue{$a_n = x^n$}, where $x$ is a constant. \pause
  \item  $a_n = x^n$ is a solution to the recurrence relation
      \[ \Blue{a_n = r_1a_{n-1} + r_{2}a_{n-2}+\cdots + r_ka_{n-k} } 
   \]
   if and only if \pause
    \[ \Blue{x^n = r_1x^{n-1} + r_{2}x^{n-2}+\cdots + r_kx^{n-k} } 
   \]
   \pause
  \item Rearranging terms leads to the \Blue{characteristic equation}:
    \[ \Blue{x^n - r_1x^{n-1} - r_{2}x^{n-2} - \cdots - r_kx^{n-k} = 0. } 
   \]
\end{itemize}
\end{frame}

\begin{frame}[plain]{}

 The \Blue{characteristic equation}:
    \[ \Blue{x^n - r_1x^{n-1} - r_{2}x^{n-2} - \cdots - r_kx^{n-k} = 0. } 
   \]
 
  {\bf Theorem 22.3}. The characteristic equation of
  the recurrence relation \Blue{$a_n = r_1a_{n-1}+r_2a_{n-2}$} is
     \[ \Blue{x^2 - r_1x - r_{2} = 0}.
     \]
     If the characteristic equation has 
     two distinct roots, $x_1$ and $x_2$, then 
     \[ \Blue{a_n = px_1^n + qx_2^n \ \ \mbox{for\ some}\ p, q}
     \]
     is the explicit formula
     for the sequence. Here, $p$ and $q$ depend on the initial conditions.
     \medskip
   
  {\bf Example 22.4}. Find an explicit formula for the sequence defined by 
    $a_n = 7a_{n-1} - 10a_{n-2}$ with $a_0=2$ and $a_1=3$.\pause
    \medskip
    
    {\bf Answer}: $a_n = \frac{7}{3}2^n - \frac{1}{3}5^n$.
%    \vspace{.3in}

\end{frame}


\begin{frame}[plain]{}

 {\bf Theorem 22.5}. If the characteristic equation 
     \[ \Blue{x^2 - r_1x - r_{2} = 0 }
     \]
     of the recurrence relation $a_n = r_1a_{n-1}+r_2a_{n-2}$ has 
     two distinct roots, $x_1$ and $x_2$, then 
     \[ \Blue{a_n = px_1^n + qx_2^n} 
     \]
     where $p$ and $q$ depend on the initial conditions, is the explicit formula
     for the sequence.
     \medskip
     
 {\bf Practice 22.6}. (Fibonacci Sequence) Solve the recurreence relation
    $f_n = f_{n-1}+f_{n-2}$ with the initial conditions $f_0=0$ and $f_1=1$. \pause
    \medskip
    
    {\bf Answer}: $f_n = \frac{\phi^n - (1-\phi)^n}{\sqrt{5}}$, where $\phi = \frac{1+\sqrt{5}}{2}$.
        (Note that $\phi\approx 1.618\ 033\ 988\ 749...$ is a golden ratio.)
    
    \vspace{.4in}
         
\end{frame}

\begin{frame}[plain]{}

 {\bf Theorem 22.7}. If the characteristic equation 
     \[ \Blue{x^2 - r_1x - r_{2} = 0 }
     \]
     of the recurrence relation $a_n = r_1a_{n-1}+r_2a_{n-2}$ has 
     a single root, $x$, then 
     \[ \Blue{a_n = px_1^n + q\Red{n}x_2^n} 
     \]
     where $p$ and $q$ depend on the initial conditions, is the explicit formula
     for the sequence.
     \medskip
     
 {\bf Practice 22.8}.  Solve the recurrence relation
    $b_n = 6b_{n-1}-9b_{n-2}$ with the initial conditions $b_0=1$ and $b_1=4$. \pause
    \medskip
    
    {\bf Answer}: $b_n = 3^n + \frac{1}{3}n3^n$
    \medskip
    
    Although we will not consider examples more complicated than
these, this characteristic root technique can be applied to much more
complicated recurrence relations. 
    
    \vspace{.2in}
         
\end{frame}

\end{document}